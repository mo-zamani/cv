\documentclass[11pt]{article}
\usepackage[dvipsnames]{xcolor}
\usepackage[left=0.75in,top=1in,right=0.75in,bottom=1in]{geometry}
\usepackage{enumitem}
\usepackage{fontawesome}
\usepackage{hyperref}
\usepackage{bookmark} % Handles outlines and resolves rerunfilecheck warning
\usepackage{graphicx}
\usepackage{fancyhdr}
\usepackage{charter} % Charter font, avoids font warnings
\usepackage{newtxmath} % Compatible math font
\usepackage{array} % For better tabular control
\usepackage{titlesec}
\usepackage{parskip}
\usepackage{setspace}

% Colors
\definecolor{headerblue}{RGB}{68, 114, 196}
\definecolor{accentblue}{RGB}{68, 114, 196}
\definecolor{sectiongray}{RGB}{0, 0, 0}
\definecolor{textgray}{RGB}{0, 0, 0}

% Page number
\pagestyle{fancy}
\fancyhf{}
\renewcommand{\headrulewidth}{0pt}
\cfoot{\thepage}

% Spacing and typography
\setlength{\parskip}{0.5em}
\raggedbottom
\newlength{\sectionspace}
\setlength{\sectionspace}{0.5em}

% Section formatting
\titleformat{\section}{\large\bfseries\color{sectiongray}}{\thesection}{1em}{}
\titlespacing*{\section}{0pt}{1.5em}{0.8em}

% Itemize formatting
\setlist[itemize]{
    label={\color{accentblue}\tiny$\bullet$},
    leftmargin=*,
    topsep=0.2em,
    parsep=0.1em,
    partopsep=0.1em,
    itemsep=0.2em
}

% Hyperlink style
\hypersetup{
    colorlinks=true,
    urlcolor=accentblue,
    linkcolor=accentblue,
    citecolor=accentblue
}

% Custom environments and commands
\newcommand{\name}[1]{{\huge\bfseries\color{headerblue} #1}\par\vspace{0.3em}}
\newcommand{\address}[1]{\centerline{\small #1}}

% Define the rSection environment
\newenvironment{rSection}[1]{
    \vspace{0.8em}
    {\large\bfseries\color{sectiongray} #1}
    \vspace{0.2em}
    \hrule height 0.2pt
    \vspace{0.5em}
}{
    \vspace{0.3em}
}

% Define the rSubsection environment
\newenvironment{rSubsection}[4]{
    \vspace{0.2em}
    \begin{minipage}[t]{0.68\textwidth}
        \raggedright
        \textbf{\color{accentblue}#1} \\ \textit{#3}
    \end{minipage}%
    \hfill%
    \begin{minipage}[t]{0.3\textwidth}
        \raggedleft
        #2 \\ \textit{#4}
    \end{minipage}
    \vspace{0.3em}
    \begin{itemize}
}{
    \end{itemize}
    \vspace{0.2em}
}

\begin{document}

% Header
\begin{center}
\name{Mohammad Zamani}
\end{center}

% Contact Information
\begin{center}
\begin{minipage}{0.95\textwidth}
    \centering
    \small
    \faMapMarker\ Tehran, Iran \quad
    \faLinkedin\ \href{https://linkedin.com/in/mohammad-zamani-087925189}{mohammad-zamani-087925189} \quad
    \faEnvelope\ mail.zamani.m@gmail.com \quad
    \faPhone\ +98 912 417 1524
\end{minipage}
\end{center}

\vspace{0.8em}

%----------------------------------------------------------------------------------------
\begin{rSection}{Education}
\begin{tabular}{@{}p{0.68\textwidth}@{}p{0.3\textwidth}@{}}
    \textbf{University of Tehran} \newline
    \textbf{M.Sc. in Structural Engineering} \newline
    School of Civil Engineering \newline
    \small{\textit{Thesis: Mathematical Modeling of Bone Fracture Healing}} &
    \raggedleft \em 2019 - 2022 \\
\end{tabular} \\
\vspace{\sectionspace}

\begin{tabular}{@{}p{0.68\textwidth}@{}p{0.3\textwidth}@{}}
    \textbf{Hekmat University} \newline
    \textbf{B.Sc. in Civil Engineering} &
    \raggedleft \em 2014 - 2017 \\
\end{tabular}
\end{rSection}

%----------------------------------------------------------------------------------------
\begin{rSection}{Technical Expertise}
\begin{tabular}{@{} l @{\hspace{6ex}} l}
\textbf{\color{accentblue}Programming} & Python (Advanced), MATLAB (Proficient) \\[0.2em]
\textbf{\color{accentblue}ML \& AI} & PyTorch, TensorFlow, Keras, Scikit-learn \\[0.2em]
\textbf{\color{accentblue}Data Science} & Pandas, NumPy, SciPy, Matplotlib, Jupyter \\[0.2em]
\textbf{\color{accentblue}Engineering Software} & Abaqus (FEA), Mathematica \\[0.2em]
\textbf{\color{accentblue}Development Tools} & LaTeX, Git, MS Office Suite
\end{tabular}
\end{rSection}

%----------------------------------------------------------------------------------------
\begin{rSection}{Research Experience}
\begin{rSubsection}{University of Tehran - HPC Lab}{2021 - Present}{Graduate Research Assistant}{}
\item Developed a novel framework for simulating bone fracture healing using FEM
\item Led comparative analysis of ML methods on engineering datasets
\item Improved multiscale homogenization with enhanced efficiency
\item Designed and tested topology optimization algorithms
\item Applied reinforcement learning to structural optimization
\end{rSubsection}

\begin{rSubsection}{Graduate Research Projects}{2019 - 2022}{University of Tehran}{}
\item Developed numerical implementations in MATLAB and Python:
\begin{itemize}
\item \textbf{Advanced FEM \& Meshless Methods}: Adaptive mesh refinement and non-linear solvers
\item \textbf{Multiscale Computing}: Homogenization, quasi-continuum methods, statistical mechanics
\item \textbf{AI Applications}: Reinforcement learning and generative AI for materials
\item \textbf{Optimization}: Gradient-based and heuristic methods
\item \textbf{Composite Materials}: Laminate theory and micromechanics
\end{itemize}
\end{rSubsection}
\end{rSection}

%----------------------------------------------------------------------------------------
\begin{rSection}{Publications}
\begin{tabular}{@{}p{0.68\textwidth}@{}p{0.3\textwidth}@{}}
    \color{accentblue}\textbf{Journal Papers} \newline
    \textbf{Published} \newline
    "Finite Element Solution of Coupled Multiphysics Reaction-Diffusion Equations for Fracture Healing in Hard Biological Tissues" &
    \raggedleft \textit{Computers in Biology and Medicine}, 2024 \\
\end{tabular} \\
\vspace{\sectionspace}

\begin{tabular}{@{}p{0.68\textwidth}@{}p{0.3\textwidth}@{}}
    \textbf{Under Review} \newline
    "The Impact of Data Splitting Methods on Machine Learning Models: A Case Study in Predicting the Concrete Workability" &
    \raggedleft 2025 \\
\end{tabular} \\
\vspace{\sectionspace}

\begin{tabular}{@{}p{0.68\textwidth}@{}p{0.3\textwidth}@{}}
    \color{accentblue}\textbf{Book Chapter} \newline
    "Biomechanics of Hard Tissues (Parts 6.1 \& 6.4)" in \textit{Multiscale Biomechanics} (S. Mohammadi) &
    \raggedleft \textit{Wiley}, 2023 \\
\end{tabular} \\
\vspace{\sectionspace}

\begin{tabular}{@{}p{0.68\textwidth}@{}p{0.3\textwidth}@{}}
    \color{accentblue}\textbf{Conference Papers} \newline
    "3D Multiscale Topology Optimization for Conceptual Design of a Quadrotor Aerial Taxi" \newline
    "Inverse Design of New Mechanical Metamaterial for Base Isolator" &
    \raggedleft \textit{ISME}, 2025 \\
\end{tabular}
\end{rSection}

%----------------------------------------------------------------------------------------
\begin{rSection}{Teaching Assistant Experience}
\begin{rSubsection}{Engineering Mathematics}{2022 -- 2024}{University of Tehran}{}
\item Held tutorials and graded assignments for undergraduate students
\end{rSubsection}

\begin{rSubsection}{Finite Element Methods}{2023 -- 2024}{University of Tehran}{}
\item Assisted students in coding assignments and FEM implementation
\end{rSubsection}

\begin{rSubsection}{Mechanics of Material II}{2021 -- 2022}{Shahid Beheshti University}{}
\item Conducted problem-solving sessions and exam preparation
\end{rSubsection}
\end{rSection}

%----------------------------------------------------------------------------------------
\begin{rSection}{References}
\begin{tabular}{@{} l @{\hspace{3em}} l}
\textbf{Dr. Soheil Mohammadi} & \textbf{Dr. Houshang Dolatshahi} \\
\textit{Full Professor, M.Sc. Supervisor} & \textit{Associate Professor} \\
University of Tehran & University of Tehran \\
smoham@ut.ac.ir & mdolat@ut.ac.ir
\end{tabular}
\end{rSection}

\end{document}